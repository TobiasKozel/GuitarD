\documentclass[a4paper,14pt]{report}
\begin{document}
\chapter{Usage \& Keyboard Shortcuts}
\begin{itemize}
  \item Dragging a Node near a connection will highlight the connection if the cursor gets close enough to it. Releasing the mouse will splice in the node into that connection.
  \item Duplicate Node: Control + Drag a Node
  \item Bypass connections of a Node: Alt + Drag a Node
  \item Quick wet/dry mix of a Node: Shift + Drag a Node
  \item Sever all connections of a Node: Alt + Control + Drag a Node
  \item Quick grab a output connection: Right-click + Drag a Node
  \item Jump to InputNode: Q
  \item Jump to OutputNode: E
  \item Jump to center: C
  \item Tidy up the node positions: F
  \item Undo: Shift + Z (DAWs might capture Control + Z strokes)
  \item Copy current preset to the clipboard: Shift + C
  \item Load preset from clipboard (Will discard the current one!): Shift + V
  \item Display copyright information about a (single) selected Node: F1
\end{itemize}

\chapter{Nodes}
\section{Feedback Node}
The only safe way to routing a signal back in the node chain.
Simply routing back a signal without it might work, but can result in undefined behavior.

\section{Graph Meta Node}
Double click it to edit the contents. Adding or deliting nodes will all take place inside this node. This also applies to loading presets.
There is no visual feedback whether you are editing a Meta Node for now.
Double clicking the background will leave the Meta Node and go back to the graph containing it (which might also be a Meta Node).
Meta Nodes can go 7 recursions deep.

\end{document}